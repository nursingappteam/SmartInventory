%Discuss the state-of-the-art with respect to your product. What solutions currently exist, and in what form (academic research, enthusiast prototype, commercially available, etc.)? Include references and citations as necessary using the \textit{cite} command, like this \cite{Rubin2012}. If there are existing solutions, why won't they work for your customer (too expensive, not fast enough, not reliable enough, etc.). This section should occupy 1/2 - 1 full page, and should include at least 5 references to related work. All references should be added to the \textit{.bib} file, fully documented in IEEE format, and should appear in the \textit{references} section at the end of this document (the IEEE citation style will automatically be applied if your reference is properly added to the \textit{.bib} file).

%ProTip: Consider using a citation manager such as Mendeley, Zotero, or EndNote to generate your \textit{.bib} file and maintain documentation references throughout the life cycle of the project.

In researching other options it appears that none of them quite match what the College of Nursing is needing. One of the more popular options that is offered by Oracle is NetSuite. This system appears to be more of an total workplace management system with more features than needed as it is more suited for a warehouse setting \cite{costello_2021}. Another related product that is being used currently was Excel. Currently the Simulation Inventory Specialist is manually tracking all of the items with a few pages. The problem with this solution is that it is very involved and time consuming. Fishbowl is another resource that could be used by the nursing department. In researching this product negative reviews where found about working with the company. This along with it's large price tag make it a poor candidate \cite{g2}. An additional Software that is close to NetSuite is Sage Intacct. Sage appears to be more of a financially focused software rather then one that keeps tabs on locations of inventory items \cite{capterra}. A software that seems to be more closely related to the needs of the nursing department is Zoho Inventory. This software is offered with a free version, so the price point matches the budget of the nursing department. But once again the functionality of this program seems to be for the selling of products rather than the tracking of where they go \cite{inventory}. Further research yielded an article by nerdwallet that provided a large list of different inventory management systems \cite{wood_2021}. All seemed to support a business that are selling their products rather then keeping tabs on their whereabouts.

From our research it appears that the niche needs of the nursing department have no Related Software that fits for them. The ability to schedule service for some assets  This shows us that for multiple reasons the nursing department has chosen to forgo other inventory management systems. 